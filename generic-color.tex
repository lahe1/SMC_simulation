% !TeX spellcheck = en_US
\documentclass[journal,twocolumn,web]{IEEEtran}

\usepackage{cite}
\usepackage{amsmath,amssymb,amsfonts}
%\usepackage[noend]{algpseudocode}
%\usepackage[ruled,vlined,linesnumbered]{algorithm2e}
\usepackage[ruled,vlined]{algorithm2e}
\usepackage{mathrsfs}
\usepackage{graphicx}
\usepackage{textcomp}
\usepackage{CJKutf8}
\usepackage{tikz} 
\usepackage{latexsym}
\usepackage{bm}
\usepackage{tikz} %Draw Automata
\usetikzlibrary{automata, positioning, arrows,shapes,snakes}
\tikzset{
	->, 
	>=stealth', 
	node distance=1.4cm, 
	every state/.style={thick,fill=blue!10}, 
	initial text=$ $, 
}
\usetikzlibrary{arrows.meta}
\newtheorem{corollary}{Corollary}
\newtheorem{notation}{Notation}
\newtheorem{theorem}{Theorem}
\newtheorem{proposition}{Proposition}
\newtheorem{lemma}{Lemma}
\newtheorem{example}{Example}
\newtheorem{definition}{Definition}
\newtheorem{remark}{Remark}
\newtheorem{property}{Property}
\newtheorem{exa}{Example}
\usepackage{subfigure}
\def\BibTeX{{\rm B\kern-.05em{\sc i\kern-.025em b}\kern-.08em
		T\kern-.1667em\lower.7ex\hbox{E}\kern-.125emX}}


\begin{document}
	\title{Supplement Material for the Paper ``Over-Approximation State Estimation for Networked Timed Discrete Event Systems with Communication Delays and Losses''}

	
	\maketitle
	
	
	

	%\end{CJK*}
	

	


	\section{Proof of Proposition 1}
	
	\begin{proposition}\label{pp1}
		Let $\mathfrak{N}=(G,oc,cc)$ be an NTDES and $G_e=(Q_e,\Sigma_e,\delta_e,q_{0_e})$ be the augmented plant for $\mathfrak{N}$. Then, the state set $Q_e$ of $G_e$ is finite.
	\end{proposition}
	\textbf{\emph{Proof}}:
		We start by considering Assumption 1, given in Eq.~(1), which states that from any state in the state set $Q$ of the original plant $G$, prior to the occurrence of a $tick$ event, no more than $|Q|-1$ events can be sent to an observation channel $oc_i$, for $i\in I_o$. Here, $|Q|$ denotes the cardinality of $Q$.  
		
		Importantly, any event transmitted through $oc_i$ may experience a delay of up to $N_{d,i}$ occurrences of the $tick$ event. Consequently, at any given time, there can be at most $N_{d,i} \cdot (|Q|-1)$ events delayed in $oc_i$. Let $\varphi$ represent the maximum delay bound across all observation channels, defined as $\varphi = \max\{N_{d,i} | i \in I_o\}$.  
		
		Since only the events in $\Sigma_{on}$ can be delayed in an observation channel, we can deduce that there are at most  
		\begin{equation*}  
			\frac{1 - (|\Sigma_{on}| \cdot (\varphi + 1))^{\varphi \cdot (|Q|-1)}}{1 - (|\Sigma_{on}| \cdot (\varphi + 1))}  
		\end{equation*}  
		possible observation channel configurations for each individual channel in $\mathfrak{N}$. This, in turn, implies that the total number of possible global observation channel configurations for $\mathfrak{N}$ is bounded by  
		\begin{equation*}  
			\left( \frac{1 - (|\Sigma_{on}| \cdot (\varphi + 1))^{\varphi \cdot (|Q|-1)}}{1 - (|\Sigma_{on}| \cdot (\varphi + 1))} \right)^k  
		\end{equation*}  
		where $k$ is the number of observation channels in $\mathfrak{N}$.  
		
		Recall that each state of the extended system $G_e$ consists of a state from $Q$ and a configuration of $A_{\Theta}$. Hence, the total number of possible states in $G_e$ is at most  
		\begin{equation*}  
			|Q| \cdot \left( \frac{1 - (|\Sigma_{on}| \cdot (\varphi + 1))^{\varphi \cdot (|Q|-1)}}{1 - (|\Sigma_{on}| \cdot (\varphi + 1))} \right)^k  
		\end{equation*}  
		Since $G$ is a finite system, it follows that the state set $Q_e$ of $G_e$ is also finite. This concludes the proof.\hfill$\blacksquare$
	
	
	

	 	
	 	
	 		\section{Proof of Theorem 1}
	 
	 	\begin{theorem}\label{T2}  
	 		Consider an NTDES $\mathfrak{N}=(G,oc,cc)$ and its augmented plant $G_e=(Q_e, \Sigma_e, \delta_e,q_{0_e})$. Let $\Gamma_S$ be the augmented supervisor w.r.t. a supervisor $S$, as defined in Eq.~(7). For any observation string $\alpha\in P_{e,o}(\mathcal{L}(\Gamma_S/\mathfrak{N}))$ generated by the compensated system $\Gamma_S/\mathfrak{N}$, $W(\alpha)=(x,\tilde{\gamma})\in I$ is an information state satisfying:  
	 		
	 		\begin{enumerate}  
	 			\item $x=E_{\Gamma_S}(\alpha)$;  
	 			\item $\tilde{\gamma}=\Gamma_S(\alpha)$.  
	 		\end{enumerate}  
	 	\end{theorem}  
	 	
	 	\textbf{\emph{Proof}}:
	 		We prove the theorem by induction on the length of strings $\alpha\in P_{e,o}(\mathcal{L}(\Gamma_S/\mathfrak{N}))$.  
	 		
	 		\textbf{Base case}: Let $|\alpha|=0$, i.e., $\alpha=\varepsilon$. In this case, the state estimate of $\Gamma_S/\mathfrak{N}$ upon observing the empty string $\varepsilon$ is $\{q_{0_{e}}\}$. Due to Eq.~(15), it holds $W(\varepsilon)=U_r(\hat{W}(\varepsilon),\Gamma_S(\varepsilon))=(x,\tilde{\gamma})$ with $x\in 2^{Q_e}$ and $\tilde{\gamma}\in 2^{\Gamma_{net}}$.   
	 		
	 		First, by Eq.~(10), we have $\tilde{\gamma}=\Gamma_S(\varepsilon)$, satisfying Statement (2). Next, we determine $x$. Due to Eq.~(9), $x$ comprises all states reachable from $\{q_{0_{e}}\}$ via unobservable events enabled by $\Gamma_S(\varepsilon)$. By Definition~5, $x$ is the state estimate of $\Gamma_S/\mathfrak{N}$ upon observing the empty string $\varepsilon$. Formally,
	 		\[    
	 		x=\{q_e\in Q_e|\exists s\in \mathcal{L}(\Gamma_S/\mathfrak{N})\cap\Sigma_{n,uo}^*:q_e=\delta_e(q_{0_e},s)\}.  
	 		\]  
	 		Due to Eq.~(8), $x=E_{\Gamma_S}(\varepsilon)$ follows, satisfying Statement (1). The base case holds.  
	 		
	 		\textbf{Induction hypothesis}: Assume that for any $\alpha\in P_{e,o}(\mathcal{L}(\Gamma_S/\mathfrak{N}))$ such that $|\alpha|\le j$, Statements (1) and (2) hold.  
	 		
	 		\textbf{Induction step}: Consider $\alpha\sigma\in P_{e,o}(\mathcal{L}(\Gamma_S/\mathfrak{N}))$ such that $|\alpha|=j$ and $\sigma\in\Sigma_{n,o}$. Write $W(\alpha\sigma)=(x,\tilde{\gamma})$ again.   
	 		
	 		First, by Eq.~(15), $W(\alpha\sigma)=U_r(\hat{W}(\alpha\sigma),\Gamma_S(\alpha\sigma))=U_r(O_r(W(\alpha),\sigma),\Gamma_S(\alpha\sigma))$ holds. It follows $\Gamma_S(\alpha\sigma)=\tilde{\gamma}$, satisfying Statement (2).  
	 		
	 		Next, to determine $x$, write $W(\alpha)=(x',\Gamma_S(\alpha))$ and $\hat{W}(\alpha\sigma)=x''$. By the induction hypothesis,  
	 		\[  
	 		x'=\{q_e\in Q_e|\exists s\in\mathcal{L}(\Gamma_S/\mathfrak{N}):P_{e,o}(s)=\alpha\land q_e=\delta_e(q_{0_e},s)\}.  
	 		\]  
	 		That it, $x'$ is the state estimate of $\Gamma_S/\mathfrak{N}$ upon observing the string $\alpha$. Based on $x'$, we determine $x''$ using Eqs.~(12)--(14). We deduce that $x''$ comprises all states reachable from a subset of $x'$ via $\sigma$ under the control of $\Gamma_S(\alpha\sigma)$. As a result, in accordance with Definition~5,  $x''$ is the state estimate of $\Gamma_S/\mathfrak{N}$ immediately after observing the string $\alpha\sigma$. Formally,  
	 		\begin{equation*}
	 			\begin{split}
	 				x''=\{q_e\in Q_e|\exists s\in&\mathcal{L}(\Gamma_S/\mathfrak{N})\cap\Sigma_e^*\Sigma_{n,o}:\\&P_{e,o}(s)=\alpha\sigma\wedge q_e=\delta_e(q_{0_e},s) \}.
	 			\end{split}
	 		\end{equation*}
	 		Finally, due to Eq.~(9), $x$ comprises all states reachable from $x''$ via unobservable events enabled by $\Gamma_S(\alpha\sigma)$. Hence, by Definition~5, $x$ is the state estimate of $\Gamma_S/\mathfrak{N}$ upon observing the string $\alpha\sigma$. We have 
	 		\[  
	 		x=\{q_e\in Q_e|\exists s\in\mathcal{L}(\Gamma_S/\mathfrak{N}):P_{e,o}(s)=\alpha\sigma\land q_e=\delta_e(q_{0_e},s)\}.  
	 		\]  
	 		Statement (1) holds. This completes the proof. \hfill$\blacksquare$
	 
	 		
	 		\section{Proof of Proposition 2}
	 		
	 		\begin{proposition}\label{nn1}
	 			Consider an NTDES $\mathfrak{N}=(G,oc,cc)$, the augmented plant $G_e$ for $\mathfrak{N}$, and a supervisor $S$. Let $\Gamma_S$ be the augmented supervisor w.r.t. $S$, as defined in Eq.~(7). For any observation string $\alpha\in P_{e,o}(\mathcal{L}(G_e))$, we have
	 		\end{proposition}
	 		\begin{equation*}\label{cbd3}
	 			\Gamma_S(\alpha) = \{\gamma \in \Gamma_{net} \,|\, \exists l \in [0, cc] : (\gamma, l) \in \Lambda_S(\alpha)\}
	 		\end{equation*}
	 		
	 		
	 		\textbf{\emph{Proof}}: 
	 			$(\subseteq)$ Consider any $\gamma\in\Gamma_{S}(\alpha)$. Due to Eq.~(7), there must exist observation strings $\alpha',\alpha''\in\Sigma_{n,o}^*$ such that $\alpha=\alpha'\alpha''$, $\#_t(\alpha'')\le cc$, and $\gamma=S(\alpha')$. According to Eq.~(16), upon observing $\alpha'$, the decision $\gamma$ paired with the maximum delay tolerance $cc$ is sent to the configuration $\Lambda_S(\alpha')$, i.e., $(\gamma,cc)\in\Lambda_S(\alpha')$.
	 			
	 			
	 			
	 			Now, as each $tick$ event in $\alpha''$ is observed, the delay tolerance decreases by one. Since there are $\#_t(\alpha'')$ such events and $\#_t(\alpha'') \leq cc$, it follows that after observing all of $\alpha''$, the remaining delay tolerance is $cc - \#_t(\alpha'')$. Therefore, $(\gamma, cc - \#_t(\alpha'')) \in \Lambda_S(\alpha'\alpha'') = \Lambda_S(\alpha)$. Because $cc - \#_t(\alpha'') \in [0, cc]$, we can conclude that 
	 				\begin{equation*}  
	 				\gamma \in \{\gamma \in \Gamma_{net} \,|\, \exists l \in [0, cc] : (\gamma, l) \in \Lambda_S(\alpha)\}
	 				\end{equation*}  
	 			
	 		
	 			$(\supseteq)$ Conversely, suppose $\gamma \in \{\gamma \in \Gamma_{net}|\exists l \in [0, cc] : (\gamma, l) \in \Lambda_S(\alpha)\}$.   
	 			Then, there exists some $l \in [0, cc]$ such that $(\gamma, l) \in \Lambda_S(\alpha)$.   
	 			By the recursive definition of $\Lambda_S$, this means that in the process of observing $\alpha$, decision $\gamma$ must have been added to the configuration at some point, and its delay tolerance was subsequently reduced to $l$.   
	 			In particular, there exists a prefix $\alpha'$ of $\alpha$ such that $\gamma = S(\alpha')$ and the pair $(\gamma, cc)$ is added to the configuration after observing $\alpha'$.   
	 			The remaining string $\alpha''$ must contain exactly $cc - l$ events of $tick$ in order to reduce the delay tolerance from $cc$ to $l$.   
	 			Therefore, $\#_t(\alpha'') = cc - l \leq cc$, satisfying the definition of augmented control decisions.   
	 			So, we can conclude that $\gamma \in \Gamma_S(\alpha)$.\hfill$\blacksquare$
	 	
	 				
	 					
	 				\section{Proof of Proposition 3}	
	 					
	 			\begin{proposition}\label{r3}
	 				Let $\mathfrak{N}=(G,oc,cc)$ be an NTDES, $G_e=(Q_e,\Sigma_e,\delta_e,q_{0_e})$ be the augmented plant for $\mathfrak{N}$, and $S$ be a supervisor. The complexity of Algorithm~1 per execution is polynomial in the number of states in $Q_e$ but exponential in the number of events in $\Sigma_e$.
	 			\end{proposition}
	 			\textbf{\emph{Proof}}: 
	 				Assuming that a string $\alpha\in P_{e,o}(\mathcal{L}(\Gamma_S/\mathfrak{N}))$ has been observed so far, and the current information state is $W(\alpha)$. Whenever a new event $\sigma\in\Sigma_{n,o}$ is observed, computing $\hat{W}(\alpha\sigma)$ involves distinguishing whether $\sigma$ is the $tick$ event. If $\sigma\ne tick$,  the operator $\hat{W}(\alpha\sigma)$ requires at most  $O(|Q_e|)$ time, where $|Q_e|$ denotes the cardinality of $Q_e$. However, if $\sigma= tick$,  the time complexity of the operator $\hat{W}(\alpha\sigma)$ is $n_1=O(2^{|\Sigma|+|\Sigma_{n,for}|}\cdot(cc+1)\cdot(|Q_e|\cdot|\Sigma|+1))$.
	 				
	 				Subsequently, computing $W(\alpha\sigma)$ in the worst case takes $n_2=O(|Q_e|\cdot|\Sigma_{n,uo}|)$ time. Thus, for each iteration of the while-loop, the algorithm has a time complexity of $O(n_1+n_2)$ for computing the current over-approximation state estimate of the NTDES $\mathfrak{N}$. 
	 				
	 				
	 				In summary, the algorithm's complexity per execution is polynomial in the number of states in $Q_e$ but exponential in the number of events in $\Sigma_e$. In practice, the number of events in $\Sigma_e$  is typically much smaller than the number of states in $Q_e$ [1]. This ensures that the algorithm can process data rapidly, making it suitable for networked engineering applications.	\hfill$\blacksquare$
	 		
	 							
	 								
	 								
	 								
	 								
	 									
	 									
	 									
	 										
	 									
	 								
	 				
	 													
	 														
	 														
	 														
	 														
	 														
	 														
	 															
	 															
	 																
	 															
	 
	
	 

	 
	 
		
\end{document}
